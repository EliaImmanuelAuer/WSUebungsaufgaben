\documentclass{article}

\usepackage[a4paper, margin=2cm]{geometry}

\usepackage{microtype}

\usepackage{enumitem}

\usepackage{array}
\usepackage{parskip}

\usepackage{amsmath}
\usepackage{amsfonts}
\usepackage{amssymb}
\usepackage{mathrsfs}
\DeclareMathAlphabet{\mathbbx}{U}{bboldx}{m}{n}
\newcommand{\mathbbxx}[1]{\pdfliteral{1 Tr 0.3 w}#1\pdfliteral{0 Tr 0 w}}

\usepackage[unicode]{hyperref}
\hypersetup{
    colorlinks,
    citecolor=black,
    filecolor=black,
    linkcolor=black,
    urlcolor=black
}
\usepackage[nameinlink]{cleveref}

\usepackage{etoolbox}
\usepackage{amsthm}
\usepackage{thmtools}
\declaretheoremstyle[
  headformat=\NAME\NOTE,
  notebraces={}{},
  notefont=\bfseries
]{mystyle}
\makeatletter
\newcommand{\HelperDeclareTheorem}[5]{
  \declaretheorem[
    name=#4,
    refname={#2, #3},
    Refname={#4, #5},
    style=mystyle
  ]{#1aux}
  \newenvironment{#1}{\begin{#1aux}\leavevmode\par}{\end{#1aux}}
}
\newcommand{\HelperDeclareTheoremWithCounter}[5]{
  \newcounter{#1}
  \declaretheorem[
    name=#4,
    refname={#2, #3},
    Refname={#4, #5},
    style=mystyle
  ]{#1aux}
  \newenvironment{#1}{\stepcounter{#1}\begin{#1aux}[\csname the#1\endcsname]\leavevmode\par}{\end{#1aux}}
}
\makeatother
\HelperDeclareTheoremWithCounter{problem}{aufgabe}{aufgaben}{Aufgabe}{Aufgaben}
\HelperDeclareTheorem{solution}{lösung}{lösungen}{Lösung}{Lösungen}

\begin{document}

\begin{problem}
Betrachten Sie den Wahrscheinlichkeitsraum $(\Omega, \mathbb{P})$ mit der Zufallsvariable $X$ gegeben durch:
\[
\begin{array}{>{\displaystyle}l}
\Omega = \{0, 1\}^3 \\
\\
\mathbb{P}: \mathcal{P}(\Omega) \to [0, 1], \\
\mathbb{P}(A) = \frac{|A|}{8} \\
\\
X: \Omega \to \mathbb{R}, \\
X \left( (a_i)_{i = 1}^3 \right) = \sum_{i = 1}^3 a_i
\end{array}
\]
Bestimmen Sie Folgendes:
\begin{itemize}
\item $\mathbb{P}(X = 0)$
\item $\mathbb{E}[X]$
\item $\operatorname{Var}[X]$
\end{itemize}
\end{problem}

\begin{solution}
-
\end{solution}

\begin{problem}
Betrachten Sie den Wahrscheinlichkeitsraum $(\Omega, \mathbb{P})$ mit der Zufallsvariable $X$ gegeben durch:
\[
\begin{array}{>{\displaystyle}l}
\Omega = \{0, 2\}^3 \\
\\
\mathbb{P}: \mathcal{P}(\Omega) \to [0, 1], \\
\mathbb{P}(A) = \frac{|A|}{8} \\
\\
X: \Omega \to \mathbb{R}, \\
X \left( (a_i)_{i = 1}^3 \right) = \sum_{i = 1}^3 \frac{1}{2} a_i
\end{array}
\]
Bestimmen Sie Folgendes:
\begin{itemize}
\item $\mathbb{P}(X = 0)$
\item $\mathbb{E}[X]$
\item $\operatorname{Var}[X]$
\end{itemize}
\end{problem}

\begin{solution}
-
\end{solution}

\begin{problem}
Betrachten Sie den Wahrscheinlichkeitsraum $(\Omega, \mathbb{P})$ mit der Zufallsvariable $X$ gegeben durch:
\[
\begin{array}{>{\displaystyle}l}
\Omega = \{1, 2\}^3 \\
\\
\mathbb{P}: \mathcal{P}(\Omega) \to [0, 1], \\
\mathbb{P}(A) = \frac{|A|}{8} \\
\\
X: \Omega \to \mathbb{R}, \\
X \left( (a_i)_{i = 1}^3 \right) = \sum_{i = 1}^3 (a_i - 1)
\end{array}
\]
Bestimmen Sie Folgendes:
\begin{itemize}
\item $\mathbb{P}(X = 0)$
\item $\mathbb{E}[X]$
\item $\operatorname{Var}[X]$
\end{itemize}
\end{problem}

\begin{solution}
-
\end{solution}

\begin{problem}
Betrachten Sie den Wahrscheinlichkeitsraum $(\Omega, \mathbb{P})$ mit der Zufallsvariable $X$ gegeben durch:
\[
\begin{array}{>{\displaystyle}l}
\Omega = \{\omega_1, \omega_2\}^3 \\
\\
\mathbb{P}: \mathcal{P}(\Omega) \to [0, 1], \\
\mathbb{P}(A) = \frac{|A|}{8} \\
\\
X: \Omega \to \mathbb{R}, \\
X \left( (a_i)_{i = 1}^3 \right) = \sum_{i = 1}^3 \begin{cases} 0, & a_i = \omega_1 \\ 1, & a_i = \omega_2 \\ \end{cases}
\end{array}
\]
Bestimmen Sie Folgendes:
\begin{itemize}
\item $\mathbb{P}(X = 0)$
\item $\mathbb{E}[X]$
\item $\operatorname{Var}[X]$
\end{itemize}
\end{problem}

\begin{solution}
-
\end{solution}

\end{document}