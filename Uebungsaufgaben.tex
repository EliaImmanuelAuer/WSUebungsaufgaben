\documentclass{article}

\usepackage[a4paper, margin=2cm]{geometry}

\usepackage{microtype}

\usepackage{enumitem}

\usepackage{array}
\usepackage{parskip}
\def\arraystretch{1.5}

\usepackage{amsmath}
\usepackage{amsfonts}
\usepackage{amssymb}
\usepackage{mathrsfs}
\DeclareMathAlphabet{\mathbbx}{U}{bboldx}{m}{n}
\newcommand{\mathbbxx}[1]{\pdfliteral{1 Tr 0.3 w}#1\pdfliteral{0 Tr 0 w}}

\usepackage[unicode]{hyperref}
\hypersetup{
    colorlinks,
    citecolor=black,
    filecolor=black,
    linkcolor=black,
    urlcolor=black
}
\usepackage[nameinlink]{cleveref}

\usepackage{etoolbox}
\usepackage{amsthm}
\usepackage{thmtools}
\declaretheoremstyle[
  headformat=\NAME\NOTE,
  notebraces={}{},
  notefont=\bfseries
]{mystyle}
\makeatletter
\newcommand{\HelperDeclareTheorem}[5]{
  \declaretheorem[
    name=#4,
    refname={#2, #3},
    Refname={#4, #5},
    style=mystyle
  ]{#1aux}
  \newenvironment{#1}{\begin{#1aux}\leavevmode\par}{\end{#1aux}}
}
\newcommand{\HelperDeclareTheoremWithCounter}[5]{
  \newcounter{#1}
  \declaretheorem[
    name=#4,
    refname={#2, #3},
    Refname={#4, #5},
    style=mystyle
  ]{#1aux}
  \newenvironment{#1}{\stepcounter{#1}\begin{#1aux}[\csname the#1\endcsname]\leavevmode\par}{\end{#1aux}}
}
\makeatother
\HelperDeclareTheoremWithCounter{problem}{aufgabe}{aufgaben}{Aufgabe}{Aufgaben}
\HelperDeclareTheorem{solution}{lösung}{lösungen}{Lösung}{Lösungen}

\begin{document}

\begin{problem}
Betrachten Sie den Wahrscheinlichkeitsraum $(\Omega, \mathbb{P})$ mit der Zufallsvariable $X$ gegeben durch:
\[
\begin{array}{>{\displaystyle}l}
\Omega = \{0, 1\}^3 \\
\mathbb{P}: \mathcal{P}(\Omega) \to [0, 1], \\
\mathbb{P}(A) = \frac{|A|}{8} \\
X: \Omega \to \mathbb{R}, \\
X \left( (a_i)_{i = 1}^3 \right) = \sum_{i = 1}^3 a_i
\end{array}
\]
Bestimmen Sie Folgendes:
\begin{itemize}
\item $\mathbb{P}(X = 0)$
\item $\mathbb{E}[X]$
\item $\operatorname{Var}[X]$
\end{itemize}
\end{problem}

\begin{solution}
Erstens gilt:
\[
\begin{array}{>{\displaystyle}r>{\displaystyle}c>{\displaystyle}l}
\mathbb{P}(X = 0) & = & \mathbb{P}(\{\omega \in \Omega \mid X(\omega) = 0\}) \\
& = & \mathbb{P} \left( \left\{ (a_i)_{i = 1}^3 \in \{0, 1\}^3 \mid X \left( (a_i)_{i = 1}^3 \right) = 0 \right\} \right) \\
& = & \mathbb{P} \left( \left\{ (a_i)_{i = 1}^3 \in \{0, 1\}^3 \;\middle|\; \sum_{i = 1}^3 a_i = 0 \right\} \right) \\
& = & \mathbb{P} \left( \left\{ (a_i)_{i = 1}^3 \in \{0, 1\}^3 \mid \forall i \in \{1, 2, 3\}: a_i = 0 \right\} \right) \\
& = & \mathbb{P}(\{(0, 0, 0)\}) \\
& = & \frac{|\{(0, 0, 0)\}|}{8} \\
& = & \frac{1}{8} \\
\end{array}
\]
Zweitens gilt:
\[
\begin{array}{>{\displaystyle}r>{\displaystyle}c>{\displaystyle}l}
\mathbb{E}[X] & = & \sum_{\omega \in \Omega} p(\omega) X(\omega) \\
& = & \sum_{(a_i)_{i = 1}^3 \in \{0, 1\}^3} p \left( (a_i)_{i = 1}^3 \right) X \left( (a_i)_{i = 1}^3 \right) \\
& = & \sum_{(a_i)_{i = 1}^3 \in \{0, 1\}^3} \frac{1}{8} X \left( (a_i)_{i = 1}^3 \right) \\
& = & \frac{1}{8} \sum_{(a_i)_{i = 1}^3 \in \{0, 1\}^3} X \left( (a_i)_{i = 1}^3 \right) \\
& = & \frac{1}{8} \sum_{(a_i)_{i = 1}^3 \in \{0, 1\}^3} \sum_{i = 1}^3 a_i \\
& = & \frac{1}{8} \sum_{(a_i)_{i = 1}^3 \in \{0, 1\}^3} (a_1 + a_2 + a_3) \\
& = & \frac{1}{8} \sum_{(a_i)_{i = 1}^3 \in \{(0, 0, 0), (0, 0, 1), (0, 1, 0), (0, 1, 1), (1, 0, 0), (1, 0, 1), (1, 1, 0), (1, 1, 1)\}} (a_1 + a_2 + a_3) \\
& = & \frac{1}{8} (0 + 1 + 1 + 2 + 1 + 2 + 2 + 3) \\[5pt]
& = & \frac{1}{8} 12 \\[5pt]
& = & \frac{12}{8} \\[5pt]
& = & \frac{3}{2} \\
\end{array}
\]
Weiterhin haben wir:
\[
\begin{array}{>{\displaystyle}l}
X^2: \Omega \to \mathbb{R}, \\
X^2 \left( (a_i)_{i = 1}^3 \right) = \left( X \left( (a_i)_{i = 1}^3 \right) \right)^2 = \left( \sum_{i = 1}^3 a_i \right)^2 \\
\end{array}
\]
Also haben wir:
\[
\begin{array}{>{\displaystyle}r>{\displaystyle}c>{\displaystyle}l}
\mathbb{E} \left[ X^2 \right] & = & \sum_{\omega \in \Omega} p(\omega) X^2 (\omega) \\
& = & \sum_{(a_i)_{i = 1}^3 \in \{0, 1\}^3} p \left( (a_i)_{i = 1}^3 \right) X^2 \left( (a_i)_{i = 1}^3 \right) \\
& = & \sum_{(a_i)_{i = 1}^3 \in \{0, 1\}^3} \frac{1}{8} X^2 \left( (a_i)_{i = 1}^3 \right) \\
& = & \frac{1}{8} \sum_{(a_i)_{i = 1}^3 \in \{0, 1\}^3} X^2 \left( (a_i)_{i = 1}^3 \right) \\
& = & \frac{1}{8} \sum_{(a_i)_{i = 1}^3 \in \{0, 1\}^3} \left( \sum_{i = 1}^3 a_i \right)^2 \\
& = & \frac{1}{8} \sum_{(a_i)_{i = 1}^3 \in \{0, 1\}^3} (a_1 + a_2 + a_3)^2 \\
& = & \frac{1}{8} \sum_{(a_i)_{i = 1}^3 \in \{(0, 0, 0), (0, 0, 1), (0, 1, 0), (0, 1, 1), (1, 0, 0), (1, 0, 1), (1, 1, 0), (1, 1, 1)\}} (a_1 + a_2 + a_3)^2 \\
& = & \frac{1}{8} (0 + 1 + 1 + 4 + 1 + 4 + 4 + 9) \\[5pt]
& = & \frac{1}{8} 24 \\[5pt]
& = & \frac{24}{8} \\[5pt]
& = & 3 \\
\end{array}
\]
Damit gilt drittens:
\[
\operatorname{Var}[X] = \mathbb{E} \left[ X^2 \right] - \mathbb{E}[X]^2 = 3 - \left( \frac{3}{2} \right)^2 = 3 - \frac{9}{4} = \frac{12}{4} - \frac{9}{4} = \frac{3}{4}
\]
\end{solution}

\newpage

\begin{problem}
Betrachten Sie den Wahrscheinlichkeitsraum $(\Omega, \mathbb{P})$ mit der Zufallsvariable $X$ gegeben durch:
\[
\begin{array}{>{\displaystyle}l}
\Omega = \{0, 2\}^3 \\
\mathbb{P}: \mathcal{P}(\Omega) \to [0, 1], \\
\mathbb{P}(A) = \frac{|A|}{8} \\
X: \Omega \to \mathbb{R}, \\
X \left( (a_i)_{i = 1}^3 \right) = \sum_{i = 1}^3 \frac{1}{2} a_i
\end{array}
\]
Bestimmen Sie Folgendes:
\begin{itemize}
\item $\mathbb{P}(X = 0)$
\item $\mathbb{E}[X]$
\item $\operatorname{Var}[X]$
\end{itemize}
\end{problem}

\begin{solution}
-
\end{solution}

\newpage

\begin{problem}
Betrachten Sie den Wahrscheinlichkeitsraum $(\Omega, \mathbb{P})$ mit der Zufallsvariable $X$ gegeben durch:
\[
\begin{array}{>{\displaystyle}l}
\Omega = \{1, 2\}^3 \\
\mathbb{P}: \mathcal{P}(\Omega) \to [0, 1], \\
\mathbb{P}(A) = \frac{|A|}{8} \\
X: \Omega \to \mathbb{R}, \\
X \left( (a_i)_{i = 1}^3 \right) = \sum_{i = 1}^3 (a_i - 1)
\end{array}
\]
Bestimmen Sie Folgendes:
\begin{itemize}
\item $\mathbb{P}(X = 0)$
\item $\mathbb{E}[X]$
\item $\operatorname{Var}[X]$
\end{itemize}
\end{problem}

\begin{solution}
-
\end{solution}

\newpage

\begin{problem}
Betrachten Sie den Wahrscheinlichkeitsraum $(\Omega, \mathbb{P})$ mit der Zufallsvariable $X$ gegeben durch:
\[
\begin{array}{>{\displaystyle}l}
\Omega = \{x_1, x_2\}^3 \\
\mathbb{P}: \mathcal{P}(\Omega) \to [0, 1], \\
\mathbb{P}(A) = \frac{|A|}{8} \\
X: \Omega \to \mathbb{R}, \\
X \left( (a_i)_{i = 1}^3 \right) = \sum_{i = 1}^3 \begin{cases} 0, & a_i = x_1 \\ 1, & a_i = x_2 \\ \end{cases}
\end{array}
\]
Bestimmen Sie Folgendes:
\begin{itemize}
\item $\mathbb{P}(X = 0)$
\item $\mathbb{E}[X]$
\item $\operatorname{Var}[X]$
\end{itemize}
\end{problem}

\begin{solution}
-
\end{solution}

\newpage

\begin{problem}
Betrachten Sie den Wahrscheinlichkeitsraum $(\Omega, \mathbb{P})$ mit der Zufallsvariable $X$ gegeben durch:
\[
\begin{array}{>{\displaystyle}l}
\Omega = \{1, 2, 3, 4, 5, 6, 7, 8\} \\
\mathbb{P}: \mathcal{P}(\Omega) \to [0, 1], \\
\mathbb{P}(A) = \frac{|A|}{8} \\
X: \Omega \to \mathbb{R}, \\
X(\omega) = \begin{cases}
0, & \omega = 1 \\
1, & \omega = 2 \\
1, & \omega = 3 \\
2, & \omega = 4 \\
1, & \omega = 5 \\
2, & \omega = 6 \\
2, & \omega = 7 \\
3, & \omega = 8 \\
\end{cases}
\end{array}
\]
Bestimmen Sie Folgendes:
\begin{itemize}
\item $\mathbb{P}(X = 0)$
\item $\mathbb{E}[X]$
\item $\operatorname{Var}[X]$
\end{itemize}
\end{problem}

\begin{solution}
-
\end{solution}

\newpage

\begin{problem}
Eine Münze wird dreimal unabhängig voneinander geworfen. Modellieren Sie dieses Experiment in einem geeigneten Wahrscheinlichkeitsraum $(\Omega, \mathbb{P})$. Geben Sie eine Zufallsvariable $X: \Omega \to \mathbb{R}$ an, die angibt, wie oft ``Kopf'' geworfen wurde.
\par
Bestimmen Sie Folgendes:
\begin{itemize}
\item $\mathbb{P}(X = 0)$
\item $\mathbb{E}[X]$
\item $\operatorname{Var}[X]$
\end{itemize}
\end{problem}

\begin{solution}
-
\end{solution}

\newpage

\begin{problem}
Sei $(\Omega, \mathbb{P})$ ein Wahrscheinlichkeitsraum und $X: \Omega \to \mathbb{R}$ eine Zufallsvariable mit $X \sim \operatorname{Bin} \left( 3, \frac{1}{2} \right)$.
\par
Bestimmen Sie Folgendes:
\begin{itemize}
\item $\mathbb{P}(X = 0)$
\item $\mathbb{E}[X]$
\item $\operatorname{Var}[X]$
\end{itemize}
\end{problem}

\begin{solution}
-
\end{solution}

\newpage

\begin{problem}
Betrachten Sie den Wahrscheinlichkeitsraum $(\Omega, \mathbb{P})$ mit der Zufallsvariable $X$ gegeben durch:
\[
\begin{array}{>{\displaystyle}l}
\Omega = \{\ast\} \\
\mathbb{P}: \mathcal{P}(\Omega) \to [0, 1], \\
\mathbb{P}(A) = |A| \\
X: \Omega \to \mathbb{R}, \\
X(\ast) = 1
\end{array}
\]
Bestimmen Sie $\mathbb{E}[X]$ und $\operatorname{Var}[X]$.
\end{problem}

\begin{solution}
-
\end{solution}

\newpage

\begin{problem}
\begin{enumerate}
\item {
Geben Sie die Definition eines Wahrscheinlichkeitsraums $(\Omega, \mathbb{P})$ an.
}
\item {
Geben Sie eine Abbildung $\mathbb{P}: \mathcal{P}(\mathbb{N}) \to [0, 1]$ an, sodass $(\mathbb{N}, \mathbb{P})$ ein Wahrscheinlichkeitsraum ist.
}
\item {
Geben Sie eine Abbildung $\mathbb{P}: \mathcal{P}(\mathbb{N}) \to [0, 1]$ an, sodass $(\mathbb{N}, \mathbb{P})$ ein Wahrscheinlichkeitsraum ist und weiterhin gilt: $\forall n \in \mathbb{N}: \mathbb{P}(\{n\}) \neq 0$.
}
\item {
Betrachten Sie den Wahrscheinlichkeitsraum $(\mathbb{N}, \mathbb{P})$ gegeben durch:
\[
\begin{array}{>{\displaystyle}l}
\mathbb{P}: \mathcal{P}(\mathbb{N}) \to [0, 1], \\
\mathbb{P}(A) = \sum_{n \in A} \frac{1}{2^n} \\
\end{array}
\]
\begin{enumerate}
\item {
Geben Sie ein Beispiel einer Zufallsvariable $X: \mathbb{N} \to \mathbb{R}_{\geq 0}$ mit $\mathbb{E}[X] < \infty$ und $\operatorname{Var}[X] < \infty$.
}
\item {
Geben Sie ein Beispiel einer Zufallsvariable $Y: \mathbb{N} \to \mathbb{R}_{\geq 0}$ mit $\mathbb{E}[Y] < \infty$ aber $\operatorname{Var}[Y] = \infty$.
}
\item {
Geben Sie ein Beispiel einer Zufallsvariable $Z: \mathbb{N} \to \mathbb{R}_{\geq 0}$ mit $\mathbb{E}[Z] = \infty$ und $\operatorname{Var}[Z] = \infty$.
}
\end{enumerate}
}
\end{enumerate}
\end{problem}

\begin{solution}
-
\end{solution}

\end{document}